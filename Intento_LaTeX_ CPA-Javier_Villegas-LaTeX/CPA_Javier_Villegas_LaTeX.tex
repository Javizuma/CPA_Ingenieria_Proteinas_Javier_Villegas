\documentclass[a4paper,11pt]{report}
\usepackage[utf8]{inputenc}
\usepackage{apacite}
\usepackage[pdftex]{graphicx}
\usepackage{subcaption}
\begin{document}
	\title{CPA:Análisis bioinformático  del virus SARS-CoV-2}
	\author{Javier Villegas Salmerón}
	\date{23 febrero de 2021}
	\maketitle
	\begin{abstract}
		La pandemia provocada por el SARS-CoV-2 ha hecho temblar los cimientos del mundo que conocemos y ha puesto en jaque a millones de personas. Por esta razón, en la asignatura de Ingeniería de proteínas del grado en Biotecnología de la UGR se nos ha asignado una proteína o varias, todas pertenecientes al virus provocador de la enfermedad conocida como COVID-19.
\\Con este trabajo quiero poner mi granito de arena en la lucha contra la enfermedad al realizar una revisión bibliográfica y a su vez un análisis bioinformático de los ficheros entregados por el profesor. A mí me fueron entregado varios de ellos, todos ellos correspondientes a cristalizaciones del RBD en unión a distintos anticuerpos, y yo he elegido para mi estudio el correspondiente al que interactúa con el anticuerpo monoclonal, o mAB, CR3022, con nombre de fichero 6yla.
\\He realizado esta elección porque me parecía la más interesante de las que he visto y porque un trabajo de todos hubiese sido demasiado extenso, por lo que he preferido enfocarme en una.
Este trabajo, junto con los códigos y ejecutables de los programas empleados, serán subidos a GitHub, una plataforma que he aprendido a utilizar a lo largo del curso en paralelo con la asignatura y que resulta muy útil debido a que permite a varias personas trabajar con un mismo código a la vez, lo que me puede resultar muy útil en el fututo, y para que esté a disposición de todo aquel que quiera consultar la información que aporto.
	\end{abstract}
 \tableofcontents
 \chapter{Revisión bibliográfica}
 \section{Introducción}
 Los coronavirus son virus envueltos cuyo genoma consiste en una única molécula de RNA monocatenario positivo o +ssRNA. \\Pertenecen a una gran familia de virus (Coronaviridae) que infectan aves y varios mamíferos, y pertenecen a los Nidovirales, denominados así debido al aspecto similar a una corona solar conferido por sus glicoproteínas de superficie.\ La familia Coronaviridae se clasifica en cuatro géneros: alfacoronavirus, betacoronavirus, deltacoronavirus y gammacoronavirus. Estos virus causan enfermedades respiratorias, gastrointestinales, neurológicas y hepáticas en varias especies, entre ellas humanos, y son capaces de mutar con rapidez, recombinarse y transmitirse de una especia a otra (zoonosis).\
Los coronavirus fueron reconocidos como causantes de serias infecciones respiratorias e intestinales después del brote del “síndrome respiratorio agudo severo” (SARS), cuyo agente etiológico emergió en China en 2002.\
El coronavirus de tipo 2 causante del síndrome respiratorio agudo severo, causante de la enfermedad por coronavirus conocida como COVID-19, o SARS-CoV-2 es un tipo de coronavirus cuya expansión mundial provocó la pandemia de COVID-19, en la que todavía nos encontramos.
A mí me ha tocado estudiar la estructura cristalina del dominio de unión al receptor del SARS-CoV-2 en complejo con distintos anticuerpos, y yo he elegido la que forma con el anticuerpo CR3022 Fab (6YLA) a la hora de realizar las actividades.\
En este trabajo realizaré una pequeña revisión de este virus que tanto ha dado que hablar en el último año, con el fin de esclarecer un poco los aspectos genéticos, moleculares y la naturaleza del SARS-2, tanto desde un punto de vista general como centrándome en los aspectos más característicos del mismo.
 \section{Bases generales SARS-COV-2}
 El SARS-CoV-2, nombrado así debido a que su secuencia genética es similar a la del SARS-CoV, es un betacoronavirus de 60 a 140 nm de diámetro, que posee una envoltura y una nucleocápside helicoidal formada por +ssRNA con cerca de 30,000 pares de bases, siendo el RNA más largo descrito en un virus.\
La estructura del virión consiste principalmente en una nucleocápside, que protege al material genético viral, y en una envoltura externa. En la nucleocápside, el genoma viral está asociado con la proteína de la nucleocápside (N), la cual, se halla fosforilada e insertada dentro de la bicapa de fosfolípidos de la envoltura externa. En cuanto a la envoltura externa, allí se encuentran proteínas estructurales principales, denominadas proteína Spike (S), proteína de membrana (M) y proteína de envoltura, además de otras proteínas accesorias, como la proteína hemaglutinina esterasa.\
Dos tercios de su material genético (ORF1a Y ORF1b), más cerca del extremo 5’, codifican para el gen de la replicasa viral. Este gen será traducido al inicio de la infección en dos poliproteínas de gran tamaño, pp1a y pp1ab. Estas posteriormente serán procesadas para generar 16 proteínas no estructurales (nsps), cuya mayoría resultan necesarias para el proceso de replicación y otras con funciones aún desconocidas; y del tercio restante se sintetiza RNA subgenómico que codifica proteínas estructurales de envoltura (E), membrana(M), nucleocápside(N) y espícula(S); junto con proteínas accesorias entre cuyas funciones destaca la evasión de la respuesta inmune del huésped.\

A continuación, voy a explicar más en profundidad dos estructuras fundamentales en la infección de SARS-CoV-2:\
 \subsection{Proteína S}

\end{document}
